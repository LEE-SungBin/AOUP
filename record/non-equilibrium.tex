%\documentclass[12pt]{article}
\documentclass[12pt]{scrartcl}
\title{Research Record (Non-equilibrium Physics)}
\nonstopmode
%\usepackage[utf-8]{inputenc}
\usepackage{graphicx} % Required for including pictures
\usepackage{array}
\usepackage[figurename=Figure]{caption}
\usepackage{float}    % For tables and other floats
\usepackage{verbatim} % For comments and other
\usepackage{amsmath}  % For math
\usepackage{amssymb}  % For more math
\usepackage{braket}   % Bracket
\usepackage{fullpage} % Set margins and place page numbers at bottom center
\usepackage{paralist} % paragraph spacing
\usepackage{listings} % For source code
\usepackage{subfig}   % For subfigures
%\usepackage{physics}  % for simplified dv, and 
\usepackage{enumitem} % useful for itemization
\usepackage{siunitx}  % standardization of si units
\usepackage{mathtools}
\usepackage{tikz,bm} % Useful for drawing plots
%\usepackage{tikz-3dplot}
\usepackage{circuitikz}
\usepackage{color}
\usepackage{kotex} % korean
\usepackage{hyperref}
\hypersetup{
    colorlinks=true,
	linktoc=all, 
    citecolor=black,
    filecolor=black,
    linkcolor=black,
    urlcolor=black
}
\DeclareMathOperator{\Tr}{Tr}
\newcommand*{\permcomb}[4][0mu]{{{}_{#3}\mkern#1#2_{#4}}}
\newcommand*{\perm}[1][-3mu]{\permcomb[#1]{P}}
\newcommand*{\comb}[1][-1mu]{\permcomb[#1]{C}}

%%% Colours used in field vectors and propagation direction
\definecolor{mycolor}{rgb}{1,0.2,0.3}
\definecolor{brightgreen}{rgb}{0.4, 1.0, 0.0}
\definecolor{britishracinggreen}{rgb}{0.0, 0.26, 0.15}
\definecolor{cadmiumgreen}{rgb}{0.0, 0.42, 0.24}
\definecolor{ceruleanblue}{rgb}{0.16, 0.32, 0.75}
\definecolor{darkelectricblue}{rgb}{0.33, 0.41, 0.47}
\definecolor{darkpowderblue}{rgb}{0.0, 0.2, 0.6}
\definecolor{darktangerine}{rgb}{1.0, 0.66, 0.07}
\definecolor{emerald}{rgb}{0.31, 0.78, 0.47}
\definecolor{palatinatepurple}{rgb}{0.41, 0.16, 0.38}
\definecolor{pastelviolet}{rgb}{0.8, 0.6, 0.79}

\numberwithin{equation}{subsubsection}

\begin{document}

\begin{center}
	\hrule
	\vspace{.4cm}
	{\textbf { \large Research Record (Non-equilibrium Physics)}}
\end{center}
\hspace*{\fill} \textbf{SungBin LEE (이성빈 $|$ 李誠彬)} \\
\hspace*{\fill} Undergraduate Student   \\
\hspace*{\fill} Dept. of Physics and Astronomy   \\
\hspace*{\fill} Seoul National University \\
\hrule

\tableofcontents % Output the table of contents (all sections on one slide)

\newpage
\section{Fall 2023 (4$^{\mathrm{th}}$ Semester)} %\hfill \newline

\subsection{August 2023}

\subsubsection{August 26th (Sat), 2023}
\begin{enumerate}
\item After discussion with professor, we decided to search for negative drag on AOUP particles in quartic polynomial external potential,
unlike linear potential which we have been doing so far. 

\item We make this decision after referring to an article which shows that the AOUP reaches equilibrium if their 
smooth interaction potential has zero third derivatives.\cite{Bonila2019}

\item The revised external potential with n$^{\mathrm{th}}$-order polynomial is as follows
\begin{align}
    \begin{cases}
        V(x) = \frac{f\Lambda}{2}\left[1-\left|\frac{2x}{\Lambda}\right|^n\right]&\text{ $|x|<\Lambda/2$}\\
        V(x) = 0 &\text{ else}
    \end{cases}
\end{align}

\item Running simulation under following parameters:
\begin{center}
    \begin{tabular}{||c|c|c|c|c|c|c|c|c|c|c|c||} 
        \hline
        \# ptcl & \# ens & bound & $\Gamma$ & T & $\tau$ & $D_a$ & $\delta t$ & init & sample & gap & order \\
        \hline\hline
        1000 & 1000 & 5.0 & 1.0 & 1.0 & 1.0 & 1.0 & 0.001 & 10,000 & 100 & 1,000 & 4\\
        \hline
    \end{tabular}
\end{center}

\begin{center}
    \begin{tabular}{||c|c|c||} 
        \hline
        slope f & lambda $\lambda$ & velocity v \\
        \hline\hline
        0.1 $\sim$ 0.5 & 0.1 $\sim$ 0.5 & 0.001 $\sim$ 10.0\\
        \hline
    \end{tabular}
\end{center}

\item Markov process is independent of history and depends only on the current status

\item The dimensionless parameter is as follows
\begin{center}
    \begin{tabular}{||c|c|c|c||} 
        \hline
        characteristic time & propulsion force & persistence length & typical velocity \\
        \hline\hline
        $\tau$ & $\Gamma\sqrt{\frac{D_a}{\tau}}$ & $\sqrt{D_a\cdot\tau}$ & $\sqrt{\frac{D_a}{\tau}}$ \\
        \hline
    \end{tabular}
\end{center}

\end{enumerate}

% \nocite{*}
\bibliographystyle{unsrt}
\bibliography{../tex/ref.bib}

\end{document}